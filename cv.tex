%%%%%%%%%%%%%%%%%%%%%%%%%%%%%%%%%%%%%%%%%
% Friggeri Resume/CV
% XeLaTeX Template
% Version 1.0 (5/5/13)
%
% This template has been downloaded from:
% http://www.LaTeXTemplates.com
%
% Original author:
% Adrien Friggeri (adrien@friggeri.net)
% https://github.com/afriggeri/CV
%
% License:
% CC BY-NC-SA 3.0 (http://creativecommons.org/licenses/by-nc-sa/3.0/)
%
% Important notes:
% This template needs to be compiled with XeLaTeX and the bibliography, if used,
% needs to be compiled with biber rather than bibtex.
%
%%%%%%%%%%%%%%%%%%%%%%%%%%%%%%%%%%%%%%%%%

\documentclass[]{friggeri-cv} % Add 'print' as an option into the square bracket to remove colors from this template for printing

\addbibresource{bibliography.bib} % Specify the bibliography file to include publications

\begin{document}

\header{edoardo}{colombo}{computing systems engineer} % Your name and current job title/field

%----------------------------------------------------------------------------------------
%   SIDEBAR SECTION
%----------------------------------------------------------------------------------------

\begin{aside} % In the aside, each new line forces a line break
\section{contact}
Via Mecenate 3/1
Milan, 20138
Italy
~
+33 7 62 30 14 83
~
\href{mailto:edo.gcolombo@gmail.com}{edo.gcolombo@gmail.com}
\href{http://edoardocolombo.me}{edoardocolombo.me}
\section{languages}
italian mother tongue
english TOEFL 108/120
french fluency
\section{programming}
{\color{red} $\varheartsuit$} Python
Javascript, Java,
CSS3 \& HTML5
\section{tools}
git, Stash, MongoDB
eclipse, \LaTeX
\end{aside}

%----------------------------------------------------------------------------------------
%   EDUCATION SECTION
%----------------------------------------------------------------------------------------

\section{education}

\begin{entrylist}
%------------------------------------------------
\entry
{2011--2014}
{Master {\normalfont in Engineering of Computing Systems}}
{Politecnico di Milano, Milan}
{110/110}
%------------------------------------------------
\entry
{2011--2014}
{Master of Science {\normalfont in Computer Science}}
{University of Illinois, Chicago}
{GPA 4/4}
%------------------------------------------------
\entry
{2013--2014}
{Visiting Student}
{Royal Holloway, University of London, Egham}
{Working on Master Thesis, Cerberus, with Prof. Lorenzo Cavallaro, from November to February.}
%------------------------------------------------
\entry
{2011--2014}
{ASP Diploma}
{Alta Scuola Politecnica, Milan}
{}
%------------------------------------------------
\entry
{2008--2011}
{Bachelor {\normalfont in Engineering of Computing Systems}}
{Politecnico di Milano, Milan}
{103/110}

%------------------------------------------------
\end{entrylist}

%----------------------------------------------------------------------------------------
%   WORK EXPERIENCE SECTION
%----------------------------------------------------------------------------------------

\section{experience}

\begin{entrylist}
%------------------------------------------------
\entry
{2014--Now}
{Amadeus}
{Antibes, France}
{\emph{Software Engineer and Tech Leader} \\
}
%------------------------------------------------
\entry
{2012-2013}
{Politecnico di Milano}
{Milan, Italy}
{\LaTeX \emph{instructor} \\
I was asked by Prof. Marco D. Santambrogio to give \LaTeX{} lessons to his biomedical engineering students.
It was a very positive and challenging experience, which let me further discover my passion for teaching.
}
%------------------------------------------------
\entry
{2012}
{Pratiche Sistemiche}
{Milan, Italy}
{\emph{Web Developer} \\
Pratiche Sistemiche is a counselling association based in Milan. I developed and currently
  maintain their website, of which the implementation required HTML5, CSS3, Javascript (jQuery)
  and PHP.
}
%------------------------------------------------
\entry
{2012}
{Turné Eventi}
{Milan, Italy}
{\emph{Chief Barman} \\
Barmen coordination and management for a carnival party held each year with thousands of attendances.
}
%------------------------------------------------
\entry
{2008}
{Energie}
{Milan, Italy}
{\emph{Shop Assistant}}
%------------------------------------------------
\end{entrylist}

%----------------------------------------------------------------------------------------
%   CONFERENCES SECTION
%----------------------------------------------------------------------------------------

\section{conferences}

\begin{entrylist}
%------------------------------------------------
\entry
{2013}
{Open Data on the Web}
{W3C, Google Campus London}
{Lightning talk: \emph{Empowering the E-government data life cycle}}
%------------------------------------------------
\end{entrylist}


%----------------------------------------------------------------------------------------
%   SCHOLARSHIPS
%----------------------------------------------------------------------------------------
\section{scholarships}
\begin{entrylist}

\entry
{2011-2012}
{Alta Scuola Politecnica scolarship}
{}
{}
\end{entrylist}

%----------------------------------------------------------------------------------------
%   MASTER THESIS
%----------------------------------------------------------------------------------------

\section{master.thesis}

\begin{entrylist}
%------------------------------------------------
\entry
{2014}
{Cerberus: Detection and Characterization of \\ Automatically-Generated Malicious Domains}
{Computer Security, Data Science}
{Botnets are networks of infected machines (the bots) controlled
by an external entity, the botmaster, who uses this infrastructure to
carry out malicious activities. The Command and Control Server is the machine
employed by the botmaster to dispatch orders to and gather data from the bots,
and the communication is established through a variety of distributed or
centralized protocols, which can vary from botnet to botnet. In the case of
DGA-based botnets, a Domain Generation Algorithm (DGA) is used to find the
\emph{rendezvous} point between the \emph{bots} and the \emph{botmaster}.
Botnets represent one of the most widespread and dangerous threats on the Internet and
therefore it is natural that researchers from both the industry and the academia
are striving to mitigate this phenomenon.
We propose Cerberus, an automated system based on machine learning, capable to automatically discover new botnets and use this
knowledge to detect and characterize malicious activities. Cerberus analyzes passive
DNS data, free of any privacy issues, which allows the system to be easily
deployable, and uses an unsupervised approach, i.e., Cerberus needs no
\emph{a priori} knowledge. In fact the system applies a series of filters to
discard legitimate domains while keeping domains generated by AGDs and likely to be malicious. Then, Cerberus
keeps record of the activity related to the IP addresses of those domains, and,
after $\Delta$ time, it is able to isolate clusters of domains belonging to the same
malicious activity. This knowledge is later used to train a classifier that will analyze
new DNS data for detection.
We tested our system in the wild by analyzing one week of real passive DNS data.
Cerberus was able to detect 47 new clusters of malicious activities: Well
known botnets as \texttt{Jadtre}, \texttt{Sality} and \texttt{Palevo} were found among the others.
Moreover the tests we ran on the classifier showed an overall accuracy of 93\%, proving
the effectiveness of the system. \\

\textbf{Advisors}\\
Prof. Lorenzo \textbf{Cavallaro}, Ph.D., \emph{Royal Holloway, University of London} \\
Prof. Federico \textbf{Maggi}, Ph.D., \emph{Politecnico di Milano} \\
Prof. Stefano \textbf{Zanero}, Ph.D., \emph{Politecnico di Milano}}
%------------------------------------------------
\end{entrylist}

%----------------------------------------------------------------------------------------
%   BACHELOR THESIS
%----------------------------------------------------------------------------------------
\section{bachelor.thesis}

\begin{entrylist}
\entry
{2011}
{L'Isola dei dinosauri}
{Software Engineering}
{As bachelor thesis, since AY 2010/2011, all of the Engineering of Computing Systems students are assigned a software engineering project. Groups of two up to three people are
made and a Java implementation featuring RMI and socket communication is required. \emph{L'Isola
dei dinosauri}(Dinosaur Island) was an adaptation from Facebook engineering puzzle Dinosaur
Island. It resulted in a multiplayer turn-based strategic game where dinosaurs had to survive in a
hostile island. \\

\textbf{Advisor} \\
Prof. Luciano \textbf{Baresi}, Ph.D., \emph{Politecnico di Milano}}
\end{entrylist}

%----------------------------------------------------------------------------------------
%   PROJECTS
%----------------------------------------------------------------------------------------
\section{projects}

\begin{entrylist}
\entry
{2011-2013}
{EMIMT}
{Alta Scuola Politecnica}
{
    E-Government meets integration and mining techniques.\\
    Alta Scuola Politecnica selects 150 students each year among the faculties of Engineering,
    Architecture and Design at Politecnico di Milano and Politecnico di Torino. Students are divided
    into multidisciplinary groups and are assigned a project. I am currently working at the design and
    implementation of a web application aiming at publishing the Italian public administrations
    balance sheets, offering data visualizations to the citizens. The  backend
    is being implemented using Sinatra and PostgreSQL.
}

\entry
{2012}
{Monster Mash}
{University of Illinois at Chicago}
{
    Design and implementation of an application to investigate the popularity of different kinds of monsters over the years in movies.
    We made use of the Internet Movie DataBase, enabling the user to filter out the data and look at the trend. The project was run on the university's Cyber Commons wall, a tiled display with touchscreen and it was implemented using Java and Processing.
}

\entry
{2012}
{Objects in the rear view mirror}
{University of Illinois at Chicago}
{
    Design and implementation of an application to investigate highway deaths from the National Highway Traffic Safety Administration, and in particular their Fatality Analysis Reporting System.
    Using Modest Maps library we plotted the geolocalized data, enabling the user to set custom filters, to visualize the trend, navigate the map. As with the previous project it was implemented
    using Java and Processing and it was run on the Cyber Commons wall.
}

\entry
{2012}
{When the wind blows}
{University of Illinois at Chicago}
{
    Design and implementation of a tool that could help a hypothetical data analyst to understand the
    epidemic spreading of a disease, by looking at the evolution of the messages written by the
    population.
    The tool made possible to add geolocalized markers on the map, customize their colors, and filter
    them by up to four keywords. To achieve speed and usability, we exploited the full text search
    functions of MySQL. The implementation, as with the two previous projects, was realized with
    Java and Processing and was run on the Cyber Commons wall.
}

\entry
{2012}
{MUSE}
{Politecnico di Milano}
{
    Morphone is a NECST Lab project for a context-aware Android-based mobile operating system. Its
    ecosystem is composed, among the others, by MPower, an adaptive power management system. I worked
    on a feedback retrieval system, able to retrieve user's feedback on MPower actions (e.g.
    diminishing screen brightness) and sending the data to a server in XML format.
}
\end{entrylist}

%----------------------------------------------------------------------------------------
%   COURSERA
%----------------------------------------------------------------------------------------
\section{data science}
\begin{entrylist}
\entry
{2014}
{The Data Scientist’s Toolbox}
{Johns Hopkins University}
{Verified Certificate https://www.coursera.org/verify/E46NGXAT5Z}

\entry
{2014}
{R Programming}
{Johns Hopkins University}
{Verified Certificate https://www.coursera.org/verify/4MD9X7DT2X}

\entry
{2014}
{Reproducible Research}
{Johns Hopkins University}
{Verified Certificate https://www.coursera.org/verify/PM8HFN4FT9}

\entry
{2014}
{Exploratory Data Analysis}
{Johns Hopkins University}
{Verified Certificate https://www.coursera.org/verify/TPR5D8DFT6}

\entry
{2014}
{Getting and Cleaning Data}
{Johns Hopkins University}
{Verified Certificate https://www.coursera.org/verify/7Y45SKR6TN}

\entry
{2014}
{Statistical Inference}
{Johns Hopkins University}
{Verified Certificate https://www.coursera.org/verify/YWWNTFAND8}
\end{entrylist}


%----------------------------------------------------------------------------------------
%   Volunteering
%----------------------------------------------------------------------------------------
\section{volunteering}
\begin{entrylist}
\entry
{2011-2012}
{Swimming assistant}
{HANDICAP...su la testa}
{
    I volunteered for HANDICAP...su la testa (HANDICAP...keep your head up), a no-profit
    organization helping people affected by various kind of mental impairments (e.g. Down syndrome, autism). My contribution consisted in assisting and playing with the disabled in a swimming pool.
}

\entry
{2008}
{Environmental Volunteer}
{American Conservation Experience}
{
    I volunteered for a no-profit organization, ``American Conservation Experience'', aiming at nature
    and wildlife preservation. Whilst our ``base'' was located in Flagstaff AZ, we were involved in
    several projects featuring Wutpaki National Park, Zion National Park and the Tucson University.
}
\end{entrylist}

\end{document}
