%%%%%%%%%%%%%%%%%%%%%%%%%%%%%%%%%%%%%%%%%
% Friggeri Resume/CV
% XeLaTeX Template
% Version 1.0 (5/5/13)
%
% This template has been downloaded from:
% http://www.LaTeXTemplates.com
%
% Original author:
% Adrien Friggeri (adrien@friggeri.net)
% https://github.com/afriggeri/CV
%
% License:
% CC BY-NC-SA 3.0 (http://creativecommons.org/licenses/by-nc-sa/3.0/)
%
% Important notes:
% This template needs to be compiled with XeLaTeX and the bibliography, if used,
% needs to be compiled with biber rather than bibtex.
%
%%%%%%%%%%%%%%%%%%%%%%%%%%%%%%%%%%%%%%%%%

\documentclass[]{friggeri-cv} % Add 'print' as an option into the square bracket to remove colors from this template for printing

\addbibresource{bibliography.bib} % Specify the bibliography file to include publications

\begin{document}

\header{edoardo}{colombo}{computing systems engineer} % Your name and current job title/field

%----------------------------------------------------------------------------------------
%   SIDEBAR SECTION
%----------------------------------------------------------------------------------------

\begin{aside} % In the aside, each new line forces a line break
\section{contact}
Via Mecenate 3/1
Milan, 20138
Italy
~
+33 7 62 30 14 83
~
\href{mailto:edo.gcolombo@gmail.com}{edo.gcolombo@gmail.com}
\href{http://edoardocolombo.me}{edoardocolombo.me}
\section{languages}
italian mother tongue
english TOEFL 108/120
french fluency
\section{programming}
{\color{red} $\varheartsuit$} Python
Javascript, Java,
CSS3 \& HTML5
\section{tools}
git, Stash, MongoDB
eclipse, \LaTeX
\end{aside}

%----------------------------------------------------------------------------------------
%   EDUCATION SECTION
%----------------------------------------------------------------------------------------

\section{education}

\begin{entrylist}
%------------------------------------------------
\entry
{2011--2014}
{Master {\normalfont in Engineering of Computing Systems}}
{Politecnico di Milano, Milan}
{110/110}
%------------------------------------------------
\entry
{2011--2014}
{Master of Science {\normalfont in Computer Science}}
{University of Illinois, Chicago}
{GPA 4/4}
%------------------------------------------------
\entry
{2013--2014}
{Visiting Student}
{Royal Holloway, University of London, Egham}
{Working on Master Thesis, Cerberus, with Prof. Lorenzo Cavallaro, from November to February.}
%------------------------------------------------
\entry
{2011--2014}
{ASP Diploma}
{Alta Scuola Politecnica, Milan}
{}
%------------------------------------------------
\entry
{2008--2011}
{Bachelor {\normalfont in Engineering of Computing Systems}}
{Politecnico di Milano, Milan}
{103/110}

%------------------------------------------------
\end{entrylist}

%----------------------------------------------------------------------------------------
%   WORK EXPERIENCE SECTION
%----------------------------------------------------------------------------------------

\section{experience}

\begin{entrylist}
%------------------------------------------------
\entry
{2014--Now}
{Amadeus}
{Antibes, France}
{\emph{Software Engineer and Tech Leader} \\
}
%------------------------------------------------
\entry
{2012-2013}
{Politecnico di Milano}
{Milan, Italy}
{\LaTeX \emph{instructor} \\
I was asked by Prof. Marco D. Santambrogio to give \LaTeX{} lessons to his biomedical engineering students.
It was a very positive and challenging experience, which let me further discover my passion for teaching.
}
%------------------------------------------------
\entry
{2012}
{Pratiche Sistemiche}
{Milan, Italy}
{\emph{Web Developer} \\
Pratiche Sistemiche is a counselling association based in Milan. I developed and currently
  maintain their website, of which the implementation required HTML5, CSS3, Javascript (jQuery)
  and PHP.
}
%------------------------------------------------
\entry
{2012}
{Turné Eventi}
{Milan, Italy}
{\emph{Chief Barman} \\
Barmen coordination and management for a carnival party held each year with thousands of attendances.
}
%------------------------------------------------
\entry
{2008}
{Energie}
{Milan, Italy}
{\emph{Shop Assistant}}
%------------------------------------------------
\end{entrylist}

%----------------------------------------------------------------------------------------
%   CONFERENCES SECTION
%----------------------------------------------------------------------------------------

\section{conferences}

\begin{entrylist}
%------------------------------------------------
\entry
{2013}
{Open Data on the Web}
{W3C, Google Campus London}
{Lightning talk: \emph{Empowering the E-government data life cycle}}
%------------------------------------------------
\end{entrylist}


%----------------------------------------------------------------------------------------
%   MASTER THESIS
%----------------------------------------------------------------------------------------

\section{thesis}

\begin{entrylist}
%------------------------------------------------
\entry
{2014}
{Cerberus: Detection and Characterization of \\ Automatically-Generated Malicious Domains}
{Computer Security, Data Science}
{Botnets are networks of infected machines (the bots) controlled
by an external entity, the botmaster, who uses this infrastructure to
carry out malicious activities. The Command and Control Server is the machine
employed by the botmaster to dispatch orders to and gather data from the bots,
and the communication is established through a variety of distributed or
centralized protocols, which can vary from botnet to botnet. In the case of
DGA-based botnets, a Domain Generation Algorithm (DGA) is used to find the
\emph{rendezvous} point between the \emph{bots} and the \emph{botmaster}.
Botnets represent one of the most widespread and dangerous threats on the Internet and
therefore it is natural that researchers from both the industry and the academia
are striving to mitigate this phenomenon.
We propose Cerberus, an automated system based on machine learning, capable to automatically discover new botnets and use this
knowledge to detect and characterize malicious activities. Cerberus analyzes passive
DNS data, free of any privacy issues, which allows the system to be easily
deployable, and uses an unsupervised approach, i.e., Cerberus needs no
\emph{a priori} knowledge. In fact the system applies a series of filters to
discard legitimate domains while keeping domains generated by AGDs and likely to be malicious. Then, Cerberus
keeps record of the activity related to the IP addresses of those domains, and,
after $\Delta$ time, it is able to isolate clusters of domains belonging to the same
malicious activity. This knowledge is later used to train a classifier that will analyze
new DNS data for detection.
We tested our system in the wild by analyzing one week of real passive DNS data.
Cerberus was able to detect 47 new clusters of malicious activities: Well
known botnets as \texttt{Jadtre}, \texttt{Sality} and \texttt{Palevo} were found among the others.
Moreover the tests we ran on the classifier showed an overall accuracy of 93\%, proving
the effectiveness of the system. \\

\textbf{Advisors}\\
Prof. Lorenzo \textbf{Cavallaro}, Ph.D., \emph{Royal Holloway, University of London} \\
Prof. Federico \textbf{Maggi}, Ph.D., \emph{Politecnico di Milano} \\
Prof. Stefano \textbf{Zanero}, Ph.D., \emph{Politecnico di Milano}}
%------------------------------------------------
\end{entrylist}

\end{document}
